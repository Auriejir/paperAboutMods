\documentclass[a4paper,11pt]{article}
\usepackage[utf8]{inputenc}
\usepackage[T1]{fontenc}
\usepackage{lmodern}
\usepackage[frenchb]{babel}
\usepackage{hyperref}
\usepackage{color}
\usepackage{graphicx}
\usepackage{shorttoc}
%\usepackage{titlepic}
\hoffset -1.1cm
\voffset -1.5cm
\textheight 23.7cm
\headheight 0.8cm
\headsep 1cm
\topmargin 0in
\textwidth 15cm
\parskip 6pt

%\newenvironement{changemargin}[2]{\begin{list}{}{%
%\setlength{\topsep}{0pt}%
%\setlength{\leftmargin}{0pt}%
%\setlength{\rightmargin{0pt}%
%\setlength{\listparindent}{\parindent}%
%\setlength{\itemindent}{\parindent}%
%\setlength{\parsep}{0pt plus 1pt}%
%\setlength{\leftmargin}{#1}%
%\setlength{\rightmargin}{#2}%
%}\item }{\end{list}}

\frenchbsetup{og=«,fg=»}

\begin{document}
\thispagestyle{empty}
\sffamily

%\changemargin{0cm}{0cm}
%\enlargethispage{8cm}
%\includegraphics{./images/page0.png}
%\author{Rutkowski Juliette - LABOLLE Victor,\\ESGI}
%\title{{\large <Mods, game experience and game development>}\\\vspace*{4cm}{\Huge ~}\\\bigskip{}\\%\bigskip{\large  }}
%\maketitle
%\vspace*{9cm}

\newpage
\shorttableofcontents{Sommaire}{1}
\newpage


\section{introduction}

\subsection{Why this document ?}

Mr Labolle’s word:
I've always loved play with my games' datas, create new things, alter others and test the limits of every single game I can lay hand on. But now I am going to work for the video game industry, I can't help but asking myself "how can I create games anyone can modify to fit his/her needs ?" and "what does it takes to create such game ?". I started those researches in order to find an answer to those two questions, hoping it would help me creating more flexible games in a first time and maybe helping other to create mod-friendly games for generations of modders.

Ms Rutkowski’s word:
My first modding experience was in 2004 with Warcraft 3, when I first started to create maps on the Age of Empire map editor. At first I just wanted to create huge battle of hundreads of units battling in an improbable area, or artistic figures aligning units. Then I discovered Warcraft III and his battle.net with tons of awesome mods which were far more enjoyable than the original game.This was also my first meeting with online gaming and gaming community. These two concept are highly related in my mind. I’m more of a consumer than a creator of modded content, and with that our work with Victor was highly profitable. We disagree on many points and that emulated our research, to prove each other our points’ worth.

\subsection{What do you mean by "mod" ?}

The word Mod is to take here as a shorthand for "game modification".

A mod is everything that alters the way the game is originally meant to be played. Said unmodified game is usually called "vanilla".

Mods can take many different forms, it can be a simple texture or a model change, the creation of any new item, character or map or even completely altering the original game to create a whole new one. When  are usually called "complete conversion"

In a result, mods can make the game much heavier than the “vanilla” version, and may require much more power from the computer. Depending of the skill of the modder, the mod is more or less optimized and can impact strongly on the performances of the game, or even badly interact with other mods.

\subsection{Different types of modes}

Mods can generally be classified in thow categories :

\begin{itemize}
\item Graphical mods: Change the visual of the game. Add new textures, shapes, models and shading effect to a game. Can also modify the user interface.
%\includegraphics[keepaspectratio=true, width=15cm]{./images/big_skyrim-deftext.jpg}
%\includegraphics[keepaspectratio=true, width=15cm]{./images/big_skyrim-hdtext.jpg}
\item Content mods: Add concrete content to a game like quests, non playing character, items, weapons, class, events, spells, races… Modify deeply the game data and is likely to be unbalanced. This modding is often banned in online game because it can be resulting in cheating (all players must be equals).
\end{itemize}

Some games use only graphical mods (mostly online games like World of Warcraft), or allow them only in local mod (you can’t play online with an overpowered weapon that can kill every one of your ennemy).

Many games allows both mods (like Minecraft that is natively displayed in a low definition cube-like texture, and can be improved from 64x64 to 1024x1024). It’s the case of most solo games (Mass Effect serie, The Elder Scrolls serie…)

When a franchise has an history of modding it can be very difficult to turn back and sell a “non modded” sequel to your 

\section{Famous Examples}

\subsection{The ones who were intended to}

\subsubsection{The Elder Scrolls serie}

The history of the Elder Scrolls’ serie’s moddability starts in 2002 with Morrowind which is the third istallment of the franchise. Upon install, you had the opportunity to install the now famous “The Elder Scrolls Construction Set” (usualy called TESCS) which allowed you to create your own content for the game. 

that simple possibility, given to the player to become not only actor but creator of his own story made his way in the mind of many players and led to the insane amount of mods this game has known since then and is still knowing. 

%\includegraphics[keepaspectratio=true, width=15cm]{./images/VG3Cs.jpg}

\subsubsection{World of Warcraft}

World of Warcraft was announced by Blizzard in September 2001, publicly released in November 2004 In USA, New Zealand and Australia, and in Korea and Europe in 2005. First players were allowed to register some macros, then full “add ons” were developped and spread in and by the community. These add ons were meant to only be “visuals”. World of Warcraft being a Massively Multiplayer Online Game, each player had to be equal in term of content, stuff, non playing character…
Many of these add ons help the player to manage his interface, to give him more information about quests, bosses, rare mobs, way to improve reputation for a certain faction… One of them, GearScore, gave a “level” to each item depending of its stats, and this add on became preponderant to rant a character, even leading to excluding some players, their stuff just weren’t good enough to take part in raid or dungeon.

In a result, when Blizzard released the biggest update for World of Warcraft, Cataclysm, in   December 2010, they included to their new “vanilla” version a lot of the most used add ons. For example, the item level (ilvl)  (used by the add on GearScore) was used as part of the gameplay. Each player was required a minimum of mean ilvl in his stuff to apply in the dungeon or raid search. It garanteed the raid leader player with enough stuff to not be a burden, and hopefully a minimum of knowledge of the strategy.

This exemple show that keeping an careful eye on mods created and loved by the community is a safe way to know how to improve your media. If most of your players want a specific interface upgrade, it can be a way to produce a better game. Also, a player who dedicate enough of his time to conceive and produce a mod is more likely to buy your next game or dlc/extension.

\subsubsection{Warcraft III}

Warcraft III Reign of Chaos, and then Frozen Throne is a real time strategy game released by Blizzard in July 2002. Even with the vanilla version, a map editor was realeased, allowing player to create their own maps, with factions, events, grounds, forest and shops entirely editable. These maps were playable in solo mode or even online with strangers. A lot of maps were created and some mods became famous and were trasnposed in other games when possible.
\begin{itemize}
\item Tower Defense : Waves of ennemy mobs spawns from one or more points and follow a determined path. Each player can build tower to defend the path and kill the monster before the reach the end point. Depending on the version the player could even build his towers in shape of a labyrinth to lengthen the path of the mobs. One “official” Tower Defense map (Defense of the Crystal Tower) was released with Frozen Throne in 2003
\item Hero Defense : Lookalike of Tower Defense. In this mod the player control a single hero, based on hero and units from the vanilla version, with stronger and more diverse abilities. This mod is really focused on team play. A variant is Enfo, with two teams of 6, where players can use spell to annoy and handicap the ennemy team. The remaining team was victorious.
\item Buy and Sell : 4 Players are sellers and try to get the better profit possible, the other players compete as buyers, trying to buy the more unit from the sellers to destroy their opponents. The buyers cannot buy any unit without the sellers . 
\item Line Wars : Two teams of opponent players compete in a small symetrical map to destroy the enemy base with the help of uncontrolable minions. Players can spend the gold earned on kills by improving their base and minions or upgrading their own stats.
\item Defense of the Ancients (Dota) : more codified and complex version of LIne Wars. Two team of five players battle in a large map with three lanes (path from one base to another) and 
a “jungle” (wild forst populated with neutral monsters) wrapping the central lane. Both base are represented by “Ancient” (animated tree of the elven faction of Warcraft).
\end{itemize}

The DotA mod was the most famous of all and was played by more than 10 millions of players. Later this mod inspired a new type of game : MOBA (Massively Online Battle Arena). Some games were created, inspired by the DotA mod : Dota2 (produced by Valve), Smite (Hi-Rez Studios), Heroes of Newerth (S2 Games), and most of all League of Legends (Riot Games). 

As an exemple,  League of legends, in the beginning of 2014, recorded more than 67 millions of monthly players and 10 millions daily players. These games participated in the soaring of the electronic sport (e-sport). Big societies like Coca Cola, Samsung, BenQ, Asus, Razer, SKT Telecom invest a lot of money in e-sport team and professional player, people win their life playing games, and the cash prizes has never been higher (more than 2 millions dollars).

And all that started with a mod designed by two young gamers in 2002.

\subsection{The ones who were not}

\subsubsection{The Half-Life series}

Initially based on the DOOM engine (but remade at 70 percent), the first half-life saw one of the most popular complete conversion still in use nowadays : counter strike.



\section{Overall benefits}

\subsection{The editor's point of view}

As an editor, allowing mods can be of a great benefit. It makes the game go beyond is vanilla version, and give the game some replay value for the player who liked the game. Many players won’t be interested in mods unless they already got a strong interest in the game, want to know/have more. For example, The Elder Scrolls serie has as a vanilla version a huge world and amount of quest and npc. Most players won’t get interested in mods unless they already feel they complete most of the game by themselves. Modders are often players a bit frustrated by a part of the game (“why doesn’t this spell exist ? I should invent it !”) and try to improve their own game experience.

The editor cannot presume of what the player will like, dislike, and what will make the player keep on playing. In extension, it’s hard to know if a player will create mods or no. Essentially, a person who already has created original content on an other game will be more likely to create on your game.

In consequences, if you provide the most complete experience while giving the possibility and tools to improve the world and go further and deeper in the world, you will gain more loyal gamers that will keep on working on the game years after your release (eg. Morrowind from the Elder Scrolls serie,still has an huge fanbase community that keep alive the spirit of the game. In extension the mod Skywind will use the graphic engine of Skyrim (TESV) to create a clone of Morrowind (TESIII)).

economicly-wise, if your game is modded, it will be played longer than average game. This means you get more time to make DLCs. however, you have to be carefull when releasing such DLC because if your modding community is good at making new complex and rich content, you will have to show them why YOU are the ones who create games for a living.


\subsection{The developer's point of view}

\subsection{The player's point of view}

As a player, each person looks for a different game experience. Some players will play only vanilla version of each game they try or buy. They are just not interested in creating content from their own imagination.

Some players will use mods, but only as consumers. They will install mods, try them, make criticism but never create by themselves. Even if they don’t contribute materially to the modding community, they keep it alive and have their interest renewed from time to time by new mods. These players are often emotionally linked to the game, share memories about it and are often loyal to the franchise. They also feel like being part of a community (“those who went beyond the original game”). It’s always rewarding as a player to feel like you are part of the “elite” or “the few ones”.

Finally, a few players will create content on your game if they can. It’s kind of an “engineer” state of mind when you “have” to understand how the game works. Theses players, when they start a game, quickly go to the step “Is this game moddable ?” and then start trying things, conceiving small add ons and enjoy the game more this way. They feel like they create their own experience of game, personnalised, that really suit them. 


There is conflict and debate among the gaming community about whereas you should play the vanilly version prior to use any add on or mods, shall they be visual or content. Some players argue than you should honor the developers’ work by playing the original version, to see really what the game actually is. It allows to fully enjoy the new content by comparing with the vanilla version.
In an other hand, players prefer to play directly with mods to have a stronger first game experience, with less bugs, 

\section{Drawbacks }

\subsection{The editor's point of view}

\subsection{The developer's point of view}

\subsection{The player's point of view}


\section{Requirements}

In this section, we will think about what does making a moddable game implies. Those rules are obviously not mandatory and many moddable games don’t follow them all but, from a modder’s point of view, they are most welcome. And you will see that you have many reasons to please your modders. 

\subsection{General advices}
Making a moddable game can follow the same rules than a group project : the file organization must be clear, documentation must be precise and, if possible, easy to access. 
the easiest to understand your organization is, the more modders you get. The more modders you get, the more content they create. The more content your game get, the more appealing it gets to new players. Easier said than done but if the job is well done, your game may still be selling after 10 years like Morrowind does.

\subsection{Architecture}
The last paragraph stated “the file organization must be clear” which, ironically, is as unclear as it can be. So, we will think about how files should / might / can be organized in order to avoid trouble in the conception phase in a first time and make your game more appealing to mod in a second time. But, for starters, let’s discuss some little details about general coding.
As many games are not done from scratch nowadays, we won’t spend days discussing game engine and how they should be done… That said, for the three remaining ones, it is important to keep in mind some little things :
\begin{itemize}
\item modding is experimenting, never say “they will never be a negative / eccessive value” negative gravity value or five billions gold are frequent so, either shield your inputs or, better, use the most generic functions you can.
\item modding is forgetting things, always think about a default file in case you can’t find the one requested. a silent sound, a default mesh, even a doNothing script… your level designer will be happy to see your defaultTrollFace.obj instead of crashing back to his desktop without any explaination.
\item modding is 90 percent testing, if something goes wrong, say it, preferably in game in a pop-up window explaining the error clearly with three buttons : “continue”, “crash” and “continue for all”. you might also put these in a log but with parsimony, those files can become huge...
\item modding is often wanting to do something new / unexpected. So, if you can think your code to be able to include add-ons, you might create the ultimate moddable engine.
\item modding should not be about learning new custom / obscure langages for every new game. when you have to script things, think about already existing langages rather than creating your own. Javascript can be an excellent choice.
\end{itemize}
Now that I defined what modding is, let’s talk about the elephant in the room… You can have a perfectly moddable game and a huge community of modders if it is a pain to install, only a few part of the ones who could have enjoyed your game with mods will try.

So, when you think about your data architecture, it is important not to make it over complicated because every file have to be found and linked directly. In a perfect world, your game editor should include a tool to create an installer.

\subsection{Files format}
this part can seem obvious for most of you but I will say it anyways : don’t use obscure / proprietary file format in your game and if you do so, think about explaining how it is organized in the documentation you would not miss to provide.
\begin{itemize}
\item Even if it seems appealing to use mp3 for sounds because it is kind of a standard, you might think about it twice since this is a proprietary format with restrictions. you moght want to try ogg which is open and free.
\item for videos, you might want to avoid mp4 which is also proprietary
\item for mesh, you might want to use obj, it is the most common format for 3D objects.
\end{itemize}

\subsection{Editor or not ?}
An editor might seems to be a time loss and a way to overcomplicate things at the beginning of your project. But, well done, it really can make things easier on the long run and, soon, you will see the benefits of itbecause it can allow every single people in your crew to participate easily to your game creation. Level designers willthen be able to work on maps on their own, directly seeing what they are doing, thus gaining time through every steps of the process.

But, your game editor should not become useless after the end of developpement. this tool took time to be completed and, if it obviously will constitute a solid base for your next game’s one, it can become so much more by beeing put in your modders hands… Allow them to use powerfull yet simple tools in order to foccus on creating content rather than fighting with protections and you will see emerge some beautifull mods which will undoubtfully greatly extend the lifespan of your game.


%\newpage
%\section{Annexes}
%\appendix
%http://www.afjv.com/news/3365_league-of-legends-les-chiffres-reveles-par-riot-games.htm
%http://uncommonculture.org/ojs/index.php/fm/article/view/2965/2526
%http://dl.acm.org/citation.cfm?id=2554089
%http://repository.cmu.edu/cgi/viewcontent.cgi?article=1009&context=etcpress
%\makeatother
%\def\@seccntformat#1{Annexe~\scname the#1\endscname:\quad}
%\makeatother
%\subsection{image}
%\label{image name}
%\includegraphics[keepaspectratio=true, width=15cm]{./images/pic.png}

\newpage
\tableofcontents
\newpage
%\paragraph{ }

\end{document}


